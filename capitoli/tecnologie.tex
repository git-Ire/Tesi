\chapter{Tecnologie}
In questo capitolo si illustrano tutte le tecnologie, software, servizi utilizzati per sviluppare le applicazioni.\\
Si preferisce realizzare un capitolo unico, sia per una panoramica maggiore, sia per evitare le ripetizioni delle descrizioni tecnologiche per ogni applicazione.
Mentre nel specifico capitolo dell'applicazione è presente un elenco di ciò che si è utilizzato.
\section{PowerPlatform} \label{tec:PowerPlatform}
PowerPlatform è una piattaforma Microsoft che racchiude vari strumenti per agevolare l’organizzazione usando strumenti innovativi e basati sul principio low-code.
Offre tantissimi modelli di qualsiasi tipologia da cui partire per creare ciò che si desidera.
Inoltre si ha molta scelta per caricare i dati sia da altri strumenti Microsoft sia da servizi esterni o anche da origini locali. \newline
Power Platform offre i seguenti software, alcuni solo online altri anche in versione dsektop dove permettono maggiori personalizzazioni.\\
Per l'illustrazione dei loghi della piattaforma si veda la \figurename \space \ref*{fig:PowerPlatform}.
\begin{figure}[h]
    \centering\includegraphics[width=\textwidth, height=\textheight,keepaspectratio]{immagini/Icone-PowerPlatform.png}
    \caption{Loghi PowerPlatform}
    \label{fig:PowerPlatform}
\end{figure}
\begin{description}
    \item \textbf{Power BI:} \label{tec:Power BI} permette l’analisi in autonomia di molti dati con parecchi strumenti per personalizzare la gestione e la visualizzazione anche mediante le funzionalità di intelligenza artificiale.
    \item \textbf{Power Apps:} \label{tec:Power Apps} permette di creare rapidamente applicazioni da zero o da modelli, il codice è necessario solo per impostare le proprietà degli elementi aggiunti.
    \item \textbf{Power Automate:} \label{tec:Power Automate} permette di automatizzare i processi organizzativi tramite l’impostazione di flussi che si attivano al seguito di un evento o in maniera ricorrente.
    \item \textbf{Power Virtual Agents:} permette di creare velocemente dei chatbot basati sull’intelligenza artificiale con anche la possibilità di usare più lingue.
    \item \textbf{Power Pages:} permette di creare rapidamente siti web, offre la possibilità di raccogliere i dati dei visitatori mediante Microsoft Dataverse.
\end{description}

\section{SharePoint} \label{tec:SharePoint}
SharePoint è una piattaforma di collaborazione, sviluppata da Microsoft, che permette di creare \glossario{intranet} tra membri dello stessa divisione, progetto agevolando la condivisione di materiale, la comunicazione, la creazione di applicazioni, siti personalizzati e liste condivise.
Inoltre permette molta personalizzazione anche per la tipologia di accesso ad ogni elemento per ogni membro; per il logo si veda la \figurename \space \ref*{fig:SharePoint}. 
In particolare SharePoint offre la possibilità di creare e condividere liste mediante il software Microsoft Lists.
\begin{figure}[H]
    \centering\includegraphics[width=0.2\textwidth, height=0.2\textheight,keepaspectratio]{immagini/logo-SharePoint.png}
    \caption{Logo SharePoint}
    \label{fig:SharePoint}
\end{figure}

\subsection{Microsoft Lists} \label{tec:Microsoft Lists}
è un servizio molto semplice e intuitivo con cui poter creare una lista di record in cui si può scegliere se visualizzare le colonne già previste di default o aggiungere nuove colonne specificando anche la tipologia di dato che verrà inserito. Per il logo si veda la \figurename \space \ref*{fig:M-Lists}.
\begin{figure}[H]
    \centering\includegraphics[width=0.2\textwidth, height=0.2\textheight,keepaspectratio]{immagini/logo-MicrosoftLists2.png}
    \caption{Logo Microsoft Lists}
    \label{fig:M-Lists}
\end{figure}

\section{Microsoft SQL} \label{tec:Microsoft SQL}
Microsoft SQL Server, prodotto da Microsoft, è uno dei RDBMS\footnote{Relation Database Management System} più diffusi al mondo. 
Utilizza una variante del linguaggio SQL\footnote{Structured Query Language} standard, Transact-SQL sviluppato da Microsoft stesso; per il logo si veda la \figurename \space \ref*{fig:M-SQL}.
\begin{figure}[H]
    \centering\includegraphics[width=0.3\textwidth, height=0.3\textheight,keepaspectratio]{immagini/logo-SqlServer.png}
    \caption{Logo Microsoft Sql Server}
    \label{fig:M-SQL}
\end{figure}

\section{DB2} \label{tec:DB2}
DB2 è un Relational Database Management System della IBM che permette di archiviare, gestire una grande mole di dati garantendo elevate prestazioni ed alta affidabilità con transazioni a bassa latenza.
Si utilizzano i comandi SQL\textsuperscript{1} per interrogare il database; per il logo si veda la \figurename \space \ref*{fig:DB2}.
\begin{figure}[H]
    \centering\includegraphics[width=0.2\textwidth, height=0.2\textheight,keepaspectratio]{immagini/logo-ibm-db2.jpg}
    \caption{Logo DB2}
    \label{fig:DB2}
\end{figure}
