\chapter{Progettazione}
\section{Applicazione per i dati del magazzino}
Il sistema aziendale \glossario{ERP} è basato su database \glossario{DB2} che si trovano sui server
aziendali in Inghilterra. Sono macchine IBM con il sistema operativo OS400, un sistema nato negli
anni ottanta in cui si dialoga sempre con uno strato software e mai direttamente con l’hardware
della macchina, ciò lo rende inattaccabile dai virus e hacker. Questo sistema si usa principalmente
per la gestione dei database, è molto veloce al suo interno ma un suo svantaggio è la lentezza
nell’estrarre i dati dall’esterno. Per connettersi dall’esterno si può usare un connettore di tipo \glossario{ODBC}
da cui si può creare una vista logica virtuale oppure si esportano i dati che vengono
salvati nei server locali; tra questi due server deve essere schedulato un aggiornamento frequente.\\
Infine Power Apps si connette direttamente al cloud di SQL Server, l’applicazione con un’unica
schermata visualizza solo le informazioni relative al codice articolo scansionato. Tutti i vari passaggi
richiedono l’accesso ai singoli database criptati mediante specifico utente e password per garantire
maggiore sicurezza.\\
Si è deciso di usare un’applicazione sviluppata mediante Power Apps di
Microsoft in quanto in azienda è il sistema principale e tutti i dipendenti, operai possiedono il
proprio account Microsoft personale con licenza completa.

\newpage

\section{Applicazione per il personale IT}
I dipendenti degli uffici information technology inseriscono le proprie attività svolte tramite il form presente nell’applicazione in \glossario{PowerApps}, ogni utente visualizza solo le proprie attività con possibilità di modificarle o eliminarle. 
Ogni attività registrata viene salvata in una lista di \glossario{Microsoft Lists}.\\
Lo scopo principale è di monitorare la produttività dell’ultimo periodo quindi si è stabilito di prevedere l’eliminazione automatica delle attività dopo un anno tramite un flusso Power Automate.
Per facilitare l’analisi della produttività e una visione a colpo d’occhio si è progettata una dashboard grafica interattiva che selezionando le singole voci permette di vedere, direttamente nel grafico, le relative distribuzioni per tipologia.

\section{Applicazione per l'inserimento richieste ordini}
I dipendenti della sede aziendale tramite il \glossario{form} nell’applicazione in \glossario{PowerApps} inseriscono la richiesta di un ordine specificando tutte le relative informazioni. 
All’inserimento della richiesta, oltre a salvarsi su una lista Microsoft, tramite \glossario{PowerAutomate} si avvia il ciclo di approvazione inviando una mail all’approvatore. 
Se approvato, oltre alla comunicazione al richiedente, automaticamente si genera la mail per il segretario burocratico che completerà la richiesta inserendo il numero d'ordine relativo e viene comunicato via mail al richiedente. 
Mentre se la richiesta viene rifiutata viene comunicato al richiedente e il ciclo termina.