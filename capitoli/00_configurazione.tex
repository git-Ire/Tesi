% PDF/A filecontents
\RequirePackage{filecontents}
\begin{filecontents*}{\jobname.xmpdata}
  \Title{Document’s title}
  \Author{Author’s name}
  \Language{it-IT}
  \Subject{The abstract, or short description.}
  \Keywords{keyword1\sep keyword2\sep keyword3}
\end{filecontents*}

\documentclass[12pt,                    % corpo del font principale
               a4paper,                 % carta A4
               twoside,                 % impagina per fronte-retro
               openright,               % inizio capitoli a destra
               english,                 
               italian,                 
               ]{book}    

%**************************************************************
% Importazione package
%************************************************************** 

\PassOptionsToPackage{dvipsnames}{xcolor} % colori PDF/A
\usepackage{colorprofiles}
\usepackage[a-2b,mathxmp]{pdfx}[2019/02/27]    % configurazione PDF/A.    validare in https://www.pdf-online.com/osa/validate.aspx
%\usepackage{amsmath,amssymb,amsthm}    % matematica
\usepackage[T1]{fontenc}                % codifica dei font: % NOTA BENE! richiede una distribuzione *completa* di LaTeX     **Io aggiungo che con MiKtex, va installato manualmente il pacchetto cm-super che è per i font.  Altrimenti altro consiglio di internet è inserire dopo qui il pacchetto {lmodern}
\usepackage[utf8]{inputenc}             % codifica di input; anche [latin1] va bene % NOTA BENE! va accordata con le preferenze dell'editor
\usepackage[english, italian]{babel}    % per scrivere in italiano e in inglese; % l'ultima lingua (l'italiano) risulta predefinita
\usepackage{bookmark}                   % segnalibri
\usepackage{caption}                    % didascalie
\usepackage{chngpage,calc}              % centra il frontespizio
\usepackage{csquotes}                   % gestisce automaticamente i caratteri (")  %guida tesi dice di opzioni [autostyle,italian=guillemets] che servono con biblatex 
\usepackage{emptypage}                  % pagine vuote senza testatina e piede di pagina
\usepackage{epigraph}			              % per epigrafi
\usepackage{eurosym}                    % simbolo dell'euro
%\usepackage{indentfirst}               % rientra il primo paragrafo di ogni sezione %sotto c'è il comando \parindent impostato a 0 pt
\usepackage{graphicx}                   % immagini
\usepackage{hyperref}                   % collegamenti ipertestuali
\usepackage[binding=7mm]{layaureo}      % margini ottimizzati per l'A4; rilegatura di 5 mm  %IRE: modificata io a 7
\usepackage{listings}                   % codici
\usepackage{microtype}                  % microtipografia
\usepackage{mparhack,fixltx2e,relsize}  % finezze tipografiche
\usepackage{nameref}                    % visualizza nome dei riferimenti                                      
\usepackage[font=small]{quoting}        % citazioni
\usepackage{subfig}                     % sottofigure, sottotabelle
\usepackage[italian]{varioref}          % riferimenti completi della pagina
\usepackage{booktabs}                   % tabelle                                       
\usepackage{tabularx}                   % tabelle di larghezza prefissata                                    
\usepackage{longtable}                  % tabelle su più pagine                                        
\usepackage{ltxtable}                   % tabelle su più pagine e adattabili in larghezza
\usepackage{setspace}                   % IRE spaziatura tra le linee, sotto è scritto il comando singlespacing, onehalfspacing, douplepspacing, libera teoricamente va racchiuso tra begin e end (spacing)(x)
\usepackage{fancyhdr}                   % IRE Extensive control of page headers and footers
\usepackage{float}                      % IRE permette di modificare il modo con cui gli ambienti tradizionali vengono composti, introducendo il concetto di stile per questi oggetti ‘galleggianti’ come immagini e tabelle.
%\usepackage{charter}                   % IRE
\usepackage{subfig}                     % IRE !!prerequisito pacchetto caption!!  permette di affiancare più figure o tabelle dando a ciascuna una sottodidascalia
%\usepackage[toc, acronym]{glossaries}   % glossario
%\usepackage[backend=biber,style=verbose-ibid,hyperref,backref]{biblatex} %bibliografia
\usepackage{tablefootnote}              % IRE pacchetto per le note in tabella

\hypersetup{
    %hyperfootnotes=false,
    %pdfpagelabels,
    %draft,	% = elimina tutti i link (utile per stampe in bianco e nero)
    colorlinks=false, %IRE cambiato
    linktocpage=true,
    pdfstartpage=1,
    pdfstartview=,
    % decommenta la riga seguente per avere link in nero (per esempio per la stampa in bianco e nero)
    %colorlinks=false, linktocpage=false, pdfborder={0 0 0}, pdfstartpage=1, pdfstartview=FitV,
    breaklinks=true,
    pdfpagemode=UseNone,
    pageanchor=true,
    pdfpagemode=UseOutlines,
    plainpages=false,
    bookmarksnumbered,
    bookmarksopen=true,
    bookmarksopenlevel=1,
    hypertexnames=true,
    pdfhighlight=/O,
    %nesting=true,
    %frenchlinks,
    %urlcolor=webbrown, IRE
    %linkcolor=RoyalBlue, IRE
    %citecolor=webgreen, IRE
    %pagecolor=RoyalBlue,
    %urlcolor=Black, linkcolor=Black, citecolor=Black, %pagecolor=Black,
    pdftitle={\myTitle},
    pdfauthor={\textcopyright\ \myName, \myUni, \myFaculty},
    pdfsubject={},
    pdfkeywords={},
    pdfcreator={pdfLaTeX},
    pdfproducer={LaTeX}
}
%\definecolor{webgreen}{rgb}{0,.5,0} IRE
%\definecolor{webbrown}{rgb}{.6,0,0} IRE
\definecolor{footer-gray}{HTML}{808080}
\onehalfspacing %IRE imposto il pacchetto setspace spaziatura tra le linee: singlespacing, onehalfspacing, douplepspacing
\parindent=0pt %IRE Rientro della prima riga.
\pagestyle{fancy}
\fancyhf{}
\rhead{\textcolor{footer-gray}{Tesi}}
%\lhead{\textcolor{footer-gray}{Sweven Team}}
\fancyfoot{}
\rfoot{\textcolor{footer-gray}{\thepage}}
%\renewcommand*\contentsname{Indice} %%IRE mi sembra inutile, era in Sweven sopra indice e paragrafi
\setcounter{tocdepth}{5}	%aggiunge paragrafi e sottoparagrafi all'indice
\setcounter{secnumdepth}{3}	%indica fino a che numero indicizazzare nell'indice con 0.0.xxx (5 indica fino a sottoparagrafi, 3 fino a subsubsection)
\newcommand{\glossario}[1]{\textit{#1}\textsubscript{\textit{G}}}  %glossario con l'indice basso G                                 
\newcommand{\myTitle}{Creazione di applicazioni low-code in ambiente Microsoft}
\newcommand{\myDegree}{Tesi di laurea}
\newcommand{\myUni}{Università degli Studi di Padova}
\newcommand{\myFaculty}{Corso di Laurea in Informatica}
\newcommand{\myDepartment}{Dipartimento di Matematica "Tullio Levi-Civita"}
\newcommand{\profTitle}{Prof.ssa }
\newcommand{\myProf}{Ombretta Gaggi}
\newcommand{\myName}{Irene Benetazzo} %correggere nel frontespizio "laureanda" se femmina
\newcommand{\myCode}{1223865}
\newcommand{\myLocation}{Padova}
\newcommand{\myAA}{2022-2023}
\newcommand{\myTime}{Febbraio 2023}

