\chapter{Introduzione}
\section{Presentazione azienda}
L'azienda Alfa\footnote{Alfa nome di fantasia  per richiesta esplicita dell'azienda} viene fondata nel 1964 in un paese nella provincia di Padova specializzandosi in prodotti per l'essiccamento e la filtrazione.
Successivamente apre altre sedi anche all'estero in Europa, negli anni novanta prima viene comprata da un'azienda inglese e dopo pochi anni da un'azienda americana.
Attualmente è una multinazionale con sede in America e sedi operative in 48 paesi di tutti i continenti. 
In provincia di Padova si è sempre lavorato principalmente prodotti per essiccamento e filtrazione che rappresentano un ramo dell’azienda, ma l’azienda produce in tantissimi ambiti: aerospaziale, elettromeccanica, climatizzazione, idraulica, pneumatica, gestione di fluidi e gas, controllo di processo, sigillatura e schermatura.
\section{Obiettivo stage}
L'obiettivo di questo stage è sviluppare due applicazioni mobile su piattaforma Microsoft che offre la possibilità di creare soluzioni low-code.
Lo scopo delle applicazioni è dare la possibilità ai dipendenti di velocizzare alcune operazioni come la visualizzazione dei dati e la registrazione delle attività svolte in giornata.

\section{Organizzazione e convenzioni del testo}
La tesi descrive tre applicazioni e di conseguenza i capitoli sono spesso suddivisi in quattro sezioni:
la prima parte comune, la seconda dedicata all'applicazione per i dati del magazzino, la terza dedicata all'applicazione per il personale dell'information technology (IT) e la quarta dedicata all'applicazione per l'inserimento richieste ordini.\\
La tesi prevede sette capitoli, di cui il primo è questo cioè l'introduzione in cui si presenta l'azienda, lo stage e i prodotti che verranno sviluppati. \\
Il \textbf{secondo capitolo} illustra le tecnologie che verranno utilizzate durante lo stage.\\
Il \textbf{terzo capitolo} identifica i requisiti delle applicazioni classificandoli in funzionali, qualitativi e di vincolo; insieme al proponente e committente si è stabilito se sono obbligatori, desiderabili o facoltativi.\\
Il \textbf{quarto capitolo} descrive la fase di progettazione delle applicazioni.\\
Il \textbf{quinto capitolo} descrive la fase di sviluppo delle applicazioni, riportando in dettaglio le parti più significative.\\
Il \textbf{sesto capitolo} descrive la fase di test ed illustra gli obiettivi raggiunti.\\
Il \textbf{settimo capitolo} contiene le considerazioni finali sul progetto di stage.\\
Nella stesura del documento sono state utilizzate le seguenti convenzioni tipografiche:
\begin{itemize}
    \item Il pedice G indica che la spiegazione di quel \glossario{termine}, scritto in corsivo, è presente nel glossario.
    \item Il numero inserito come apice indica che è presente la relativa nota a fine pagina.
    \item Il codice e le formule riportate verranno scritte mediante il seguente \texttt{font dattilografico} seguite dalla descrizione in \textit{corsivo}.    
\end{itemize}

\section{Applicazione per i dati del magazzino}
\subsection{Situazione iniziale}
Gli operatori di magazzino con la semplice necessità di visualizzare alcuni dati, come la locazione di un determinato articolo, devono necessariamente usare un computer per accedere al sistema aziendale \glossario{ERP} che è implementato senza un’interfaccia grafica. 
Inoltre l’operatore in magazzino non ha un suo computer personale ma è presente soltanto qualche postazione fissa di computer.
\subsubsection{Enterprise Resource Planning (ERP)}
ERP è un sistema che si occupa della gestione e pianificazione aziendale delle risorse integrando tutti i moduli aziendali: amministrazione, contabilità, produzione, magazzino, logistica, acquisti, vendite, etc…
L’avere tutto insieme con una sincronizzazione continua ben organizzata incrementa la produttività, ottimizza la gestione dei materiali e le fasi di produzione, agevolando anche il coordinamento.

\subsection{Obiettivo}
L'obiettivo è facilitare, modernizzare e velocizzare l’accesso alla visualizzazione dei dati creando un’applicazione installabile in qualsiasi dispositivo mobile (smartphone o tablet). 
In azienda è già molto utilizzato l’ambiente Microsoft, quindi si è pensato di utilizzare PowerPlatform, in particolare Power Apps.


\section{Applicazione per il personale IT}
\subsection{Situazione iniziale}
I dipendenti del personale information technology non utilizzavano nessuna applicazione, form o altro metodo per registrare le attività svolte durante la giornata di lavoro.
\subsection{Obiettivo}
L'obiettivo è fornire una semplice applicazione, usabile da qualsiasi dispositivo elettronico, per compilare il form e registrare le proprie attivià svolte o visualizzare i dati inseriti avendo la possibilità di modificarli o eliminarli; inoltre, i dirigenti, desiderano avere la visualizzazione grafica dei dati.


\section{Applicazione per l'inserimento richieste ordini}
\subsection{Situazione iniziale}
I dipendenti della sede aziendale inserivano la richiesta dell'ordine direttamente nella lista, avvisavano manualmente tramite mail l'approvatore e il segretario che la processava burocraticamente. Inoltre avendo libero accesso alla lista tutto era visibile e modificabile da tutti.
\subsection{Obiettivo}
L'obiettivo è creare un'applicazione per agevolare e modernizzare l'inserimento della richiesta, creare un ciclo per automatizzare le varie fasi per l'approvazione della richiesta di un'ordine.

