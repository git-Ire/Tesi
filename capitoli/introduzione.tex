\chapter{Introduzione}
\section{Presentazione azienda}
L'azienda Alfa\footnote{Alfa nome di fantasia  per richiesta esplicita dell'azienda} viene fondata nel 1964 in un paese nella provincia di Padova specializzandosi in prodotti per l'essiccamento e la filtrazione.
Successivamente apre altre sedi anche all'estero in Europa, negli anni novanta prima viene comprata da un'azienda inglese e dopo pochi anni da un'azienda americana.
Attualmente è una multinazionale con sede in America e sedi operative in 48 paesi di tutti i continenti. 
In provincia di Padova si è sempre lavorato principalmente prodotti per essiccamento e filtrazione che rappresentano un ramo dell’azienda, ma l’azienda produce in tantissimi ambiti: aerospaziale, elettromeccanica, climatizzazione, idraulica, pneumatica, gestione di fluidi e gas, controllo di processo, sigillatura e schermatura.
\section{Obiettivo stage}
L'obiettivo di questo stage è sviluppare due applicazioni mobile su piattaforma Microsoft che offre la possibilità di creare soluzioni low-code.
Lo scopo delle applicazioni è dare la possibilità ai dipendenti di velocizzare alcune operazioni come la visualizzazione dei dati e la registrazione delle attività svolte in giornata.

\newpage
\section{Organizzazione del testo}
La tesi descrive quattro applicazioni ed è sviluppata in sette capitoli:
\begin{description}
  \item[Primo capitolo] introduce presentando l'azienda, lo stage e i prodotti sviluppati.
  \item[Secondo capitolo] illustra le tecnologie che verranno utilizzate durante lo stage per lo sviluppo dei prodotti.
  \item[Terzo capitolo] descrive in dettaglio l'applicazione per i dati del magazzino.
  \item[Quarto capitolo] descrive in dettaglio l'applicazione per il personale dell'ufficio information technology (IT).
  \item[Quinto capitolo] descrive in dettaglio l'applicazione per l'inserimento richieste ordini da parte dei dipendenti aziendali.
  \item[Sesto capitolo] descrive in dettaglio l'applicazione per le segnalazioni della produzione.
  \item[Settimo capitolo] contiene le considerazioni finali sul progetto di stage.    
\end{description}
In particolare il terzo, quarto, quinto e sesto capitolo sono ugualmente impostati con al loro interno quattro sezioni: analisi dei requisiti, progettazione, sviluppo, verifica e validazione.

\subsection{Convenzioni tipografiche}
Nella stesura del documento sono state utilizzate le seguenti convenzioni tipografiche:
\begin{itemize}
    \item Il pedice G indica che la spiegazione di quel \glossario{termine}, scritto in corsivo, è presente nel glossario.
    \item Il numero inserito come apice indica che è presente la relativa nota a fine pagina.
    \item Il codice e le formule riportate verranno scritte mediante il seguente \texttt{font dattilografico} seguite dalla descrizione in \textit{corsivo}.    
\end{itemize}

\newpage
\subsubsection{Legenda Analisi dei Requisiti}
Le richieste dei proponenti per le applicazioni sono state suddivise in requisiti funzionali, qualitativi o di vincolo; inoltre sono stati classificati in obbligatori, desiderabili o facoltativi.
La classificazione dei requisiti verrà identificata tramite il seguente codice che viene descritto nella \tablename \space \ref*{tab:Requisiti}.
\begin{center}
  \textbf{R[TIPO][PRIORITA'][NUMERO]-[APPLICAZIONE]}
\end{center}
\renewcommand{\arraystretch}{1.5} %ampiezza righe
\begin{table}[bh]
\begin{tabular}{ |m{8em}|m{26em}| }
  \hline
  \textbf{Nome} & \textbf{Descrizione} \\
  \hline
  R & Acronimo di Requisito \\
  \hline
  TIPO & Indica il tipo di requisito: \\
        & \textbf{F}: Requisto funzionale, definizione di una caratteristica necessaria nel software \\
        &	\textbf{V}: Requisito di vincolo, rappresenta un vincolo avanzato \\
        &	\textbf{Q}: Requisito di qualità, inerente le regole di qualità \\
  \hline
  PRIORITA' & Indica il tipo di priorità: \\
        &	\textbf{O}: Requisito obbligatorio \\
        &	\textbf{D}: Requisito desiderabile \\
        &	\textbf{F}: Requisito facoltativo \\
  \hline
  NUMERO & Codice Numerico Identificativo \\
  \hline
  APPLICAZIONE & Indica per quale applicazione: \\
               & \textbf{M}: Applicazione per i dati del magazzino \\
               & \textbf{IT}: Applicazione per il personale information technology \\
               & \textbf{OR}: Applicazione per l'inserimento richieste ordini \\
               & \textbf{P}: Applicazione per le segnalazioni della produzione \\
  \hline
\end{tabular}
\caption{Legenda classificazione requisiti}
\label{tab:Requisiti}
\end{table}

\newpage
\section{Applicazione per i dati del magazzino}
\subsection{Situazione iniziale}
Gli operatori di magazzino con la semplice necessità di visualizzare alcuni dati, come la locazione di un determinato articolo, devono necessariamente usare un computer per accedere al sistema aziendale \glossario{ERP} che è implementato senza un’interfaccia grafica. 
Inoltre l’operatore in magazzino non ha un suo computer personale ma è presente soltanto qualche postazione fissa di computer.
\subsubsection{Enterprise Resource Planning (ERP)} \label{tec:ERP}
ERP è un sistema che si occupa della gestione e pianificazione aziendale delle risorse integrando tutti i moduli aziendali: amministrazione, contabilità, produzione, magazzino, logistica, acquisti, vendite, etc…
L’avere tutto insieme con una sincronizzazione continua ben organizzata incrementa la produttività, ottimizza la gestione dei materiali e le fasi di produzione, agevolando anche il coordinamento.

\subsection{Obiettivo}
L'obiettivo è facilitare, modernizzare e velocizzare l’accesso alla visualizzazione dei dati creando un’applicazione installabile in qualsiasi dispositivo mobile (smartphone o tablet). 
In azienda è già molto utilizzato l’ambiente Microsoft, quindi si è pensato di utilizzare \glossario{PowerPlatform}, in particolare \glossario{Power Apps}.


\section{Applicazione per il personale IT}
\subsection{Situazione iniziale}
I dipendenti del personale information technology non utilizzavano nessuna applicazione, form o altro metodo per registrare le attività svolte durante la giornata di lavoro.
\subsection{Obiettivo}
L'obiettivo è fornire una semplice applicazione, usabile da qualsiasi dispositivo elettronico, per compilare il form e registrare le proprie attivià svolte o visualizzare i dati inseriti avendo la possibilità di modificarli o eliminarli; inoltre, i dirigenti, desiderano avere la visualizzazione grafica dei dati.


\section{Applicazione per l'inserimento richieste ordini}
\subsection{Situazione iniziale}
I dipendenti della sede aziendale inserivano la richiesta dell'ordine direttamente nella lista, avvisavano manualmente tramite mail l'approvatore e il segretario che la processava burocraticamente. Inoltre avendo libero accesso alla lista tutto era visibile e modificabile da tutti.
\subsection{Obiettivo}
L'obiettivo è creare un'applicazione per agevolare e modernizzare l'inserimento della richiesta, creare un ciclo per automatizzare le varie fasi per l'approvazione della richiesta di un'ordine.


\section{Applicazione per segnalazioni della produzione}
\subsection{Situazione iniziale}
Gli operai della sede aziendale quando necessitavano di fare una segnalazione di mancanza di alcuni componenti comunicavano direttamente ai loro capi, a voce o mail, spesso senza fornire i dettagli completi sul prodotto e le informazioni per agevolare la fornitura.\\
Il report complessivo delle linee di produzione era un solo grafico realizzato in Excel, molto scarno cioè con solo riportato il numero di pezzi versati in magazzino, spesso inesatto e non si aggiornava automaticamente.
\subsection{Obiettivo}
L'obiettivo è creare un'applicazione per agevolare le segnalazioni dei componenti mancanti o del tempo perso in extra ciclo su una linea di produzione per un determinato problema.
L'invio automatico ai responsabili della mail con tutte le relative informazioni al fine di ottimizzare i tempi per la risoluzione del problema o fornitura dei componenti mancanti.\\
Un altro obiettivo è realizzare una dashboard che riassume lo stato di produttività giornaliero delle linee di produzione con dati corretti e aggiornati frequentemente in modo automatico.
