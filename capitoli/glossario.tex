\chapter{Glossario}
\begin{description}
    \item[Dashboard] in italiano si traduce cruscotto; raccoglie un insieme di dati, grafici, liste o tabelle che a colpo d'occhio forniscono un monitor sull'andamento di ciò che si sta analizzando.
    \item[DAX] Data Analysis Expressions è un linguaggio costituito da funzioni, operatori e costanti che si usano nelle formule o espressioni nei servizi di analisi dati per eseguire query e calcoli nei modelli di dati tabulari.
    \item[DB2] è un Relational Database Management System della IBM, si rimanda alla descrizione nella relativa sezione \ref*{tec:DB2}.
    \item[ERP] Enterprise Resource Planning è un sistema che si occupa della gestione e pianificazione aziendale delle risorse; si rimanda alla descrizione nella relativa sezione \ref*{tec:ERP}.
    \item[IBM] è l'azienda informatica più anziana, attiva già dalla fine dell'Ottocento, ed è tra le più grandi al mondo nel settore informatico.
    \item[IBM AS400] computer sviluppati dall'azienda IBM, commercializzati nel 1988, è un calcolatore che può servire migliaia di utenti contemporaneamente nell'esecuzione di programmi di gestione aziendale. Utilizza il sistema operativo OS400
    \item[Intranet]  è una rete privata aziendale che agevola la comunicazione interna consentendo di collaborare e semplificare i processi organizzativi.
    \item[Join] operazione tipica nei database, cioè unire due tabelle diverse. Esistono più tipologie di join e può essere imposta una condizione come filtro.
    \item[Linked Server] collegamento che permette di accedere ai dati gestiti da SQL Server o da altre sorgenti dati.
    \item[Microsoft Lists] è un software di Microsoft che permette di creare una lista di record, si rimanda alla descrizione nella relativa sezione \ref*{tec:Microsoft Lists}.
    \item[Microsoft SQL] è un Relational Database Management System di Microsoft, si rimanda alla descrizione nella relativa sezione \ref*{tec:Microsoft SQL}
    \item[ODBC] Open Database Connectivity è uno standard di Microsoft per permettere l'accesso e lo scambio dei dati.
    \item[OS400] sistema operativo sviluppato appositamente per i computer IBM AS400. Sistema operativo ad oggetti con già integrato un database DB2.
    \item[PowerPlatform] è la piattaforma sviluppata da Microsoft, basata sul principio low-code, affinchè gli utenti possano creare rapidamente e senza esserne esperti applicazioni, siti, report, automatizzare i processi; si rimanda alla descrizione nella relativa sezione \ref*{tec:PowerPlatform}
    \item[Power Apps] è un software di Microsoft, appartenente a PowerPlatform, che permette di realizzare applicazioni per tutti i dispositivi; si rimanda alla descrizione nella relativa sezione \ref*{tec:Power Apps}
    \item[Power Automate] è un servizio di Microsoft, appartenente a PowerPlatform, che permette di creare l'automazione dei processi organizzativi; si rimanda alla descrizione nella relativa sezione \ref*{tec:Power Automate}
    \item[Power Bi] è un software di Microsoft, appartenente a PowerPlatform, che realizza grafici per favorirne l'analisi; si rimanda alla descrizione nella relativa sezione \ref*{tec:Power BI} 
    \item[Query] interrogazione che viene svolta sul database al fine di estrapolare informazioni.
    \item[Record] è un oggetto composto da più campi di tipo eterogeneo tra loro, in ambito database e liste racchiude i campi di un elemento.
    \item[SharePoint] è una piattaforma di collaborazione, di Microsoft, permette di creare intranet; si rimanda alla descrizione nella relativa sezione \ref*{tec:SharePoint}.
\end{description}

